

\begin{frame}
    \frametitle{Interfejsy standardowe}
    Interfejsy zapewniające definicje metod 
    przydatne przy implementacji serwisów.
    Bindingi mogą implicite implementować!
    \begin{itemize}
        \item \href{https://dbus.freedesktop.org/doc/dbus-java/api/org/freedesktop/DBus.Peer.html}{org.freedesktop.DBus.Peer} - metody Ping i GetMachineId \pause
        \item \href{https://dbus.freedesktop.org/doc/dbus-java/api/org/freedesktop/DBus.Introspectable.html}{org.freedesktop.DBus.Introspectable} - reprezentacja obiektu w XML \pause
        \item \href{https://dbus.freedesktop.org/doc/dbus-java/api/org/freedesktop/DBus.Properties.html}{org.freedesktop.DBus.Properties} - manipulacja właściwościami, odczyt wartości \pause
        \item \href{URL}{org.freedesktop.DBus.ObjectManager} - zarządzanie obiektami udostępnianymi przez serwis
    \end{itemize}
\end{frame}

%\begin{frame}
%    \frametitle{Co możemy implementować?}
%    Istnieje wiele interfejsów z możliwością implementacji.
%    Przykładowe API do implementacji:
%    \begin{itemize}
%        \item Interfejsy standardowe
%        \item \href{https://specifications.freedesktop.org/mpris-spec/latest/}{Standard MPRIS}-odtwarzacze muzyki
%    \end{itemize}
%\end{frame}

\begin{frame}[fragile]
    \frametitle{XML}
    \lstset{
    basicstyle=\tiny\ttfamily, 
    breaklines=true,           
    breakatwhitespace=true,    
    stepnumber=1               
}
    \begin{lstlisting}
    <node>
        <interface name="org.meks.Logger">
            <method name="AddLog">
                <arg direction="in"  type="s" name="log_str" />
                <arg direction="out" type="s" />
            </method>
            <signal name="CountChange">
                <arg direction="out" type="i" name="count" />
            </signal>
            <property name="Upper" type="b" access="readwrite"/>
            <property name="LogCount" type="i" access="read" />
            <property name="LogFileName" type="s" access="readwrite"/>
        </interface>
    </node>

    \end{lstlisting}
\end{frame}


\begin{frame}
    \frametitle{Policy XML}
    Kontrola dostępu do serwisów, metod i sygnałów.
    Znajduje się w /etc/dbus-1/ lub /usr/share/dbus-1
\end{frame}



\begin{frame}
    \frametitle{Narzędzia}
    \begin{itemize}
        \item dbus-monitor - debugger do monitorowania wiadomości dbusa
        \item dbus-send - wysyłanie wiadomości do busy
        \item d-feet - graficzny debugger do sprawdzania
        serwisów i wysyłania wiadomości
        \item 
    \end{itemize}
\end{frame}

\begin{frame}
    \frametitle{Implementacje, Bindingi}
    Freedesktop.org oferuje implementacje i bindingi protokołu w D-Bus. Najciekawsze:
    \begin{itemize}
        \item pydbus - wysokopoziomowa implementacja w pythonie, nie mylić z dbus library!
        \item zbus - implementacja w rust
        \item libdbus - niskopoziomowe API w C, nie polecane do pisania 
        prostych aplikacji
        \item GDbus - implementacja w C, już lepiej w tym pisać.
        Ale kod lepiej wygenerować
    \end{itemize}
\end{frame}