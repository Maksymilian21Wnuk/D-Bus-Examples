\begin{frame}
    \frametitle{D-Bus}
    Potrzeba zaimplementowania ustandaryzowanego,
    bezpiecznego IPC dla
    środowisk graficznych.

    2002 - początek projektu

    2006 - stabilna wersja
\end{frame}

\begin{frame}
    \frametitle{Działanie}
    \includegraphics[width=\textwidth,height=\textheight,keepaspectratio]{dbus.png}
\end{frame}

\begin{frame}
    \frametitle{Zalety}
    \begin{itemize}
        \item Relatywnie prosty (ale nie libdbus...)
        \item Dostępność standardu w wielu językach programowania
        \item Security policy
        \item 
    \end{itemize}
\end{frame}

\begin{frame}
    \frametitle{D-Bus bus}
    Demon do którego łączą się aplikacje. Przekierowuje
    wiadomości od aplikacji do innych aplikacji. Zaimplementowany
    w libdbus.
\end{frame}

\begin{frame}
    \frametitle{libdbus}
    Niskopoziomowa biblioteka pozwalająca aplikacjom na wymianę 
    wiadomości.
\end{frame}

\begin{frame}
    \frametitle{System bus}
    Z poziomu użytkownika, osobna instancja dla każdego.
    \begin{itemize}
        \item GUI KDE, GNOME
        \item Aplikacje, Spotify, Firefox
        \item PulseAudio
    \end{itemize}
\end{frame}

\begin{frame}
    \frametitle{Session bus}
    Z poziomu systemu operacyjnego.
    \begin{itemize}
        \item Sieci, NetworkManager
        \item Urządzenia, UDisks, USB
        \item Uprawnienia, PolicyKit
    \end{itemize}
\end{frame}

\begin{frame}
    \frametitle{Połączenie serwisu}
    Następuje przy połączeniu do demona.
    Serwis ma przyznawaną nazwę:
    \begin{itemize}
        \item Unikalna: :1-37, :1-42
        \item Well-known name: org.Cinammon, org.mpris.MediaPlayer2.spotify
        \item Znak : zarezerwowany
    \end{itemize}
\end{frame}

\begin{frame}
    \frametitle{Obiekt}
    \begin{itemize}
        \item Ścieżka do obiektu jako nazwa: 
        \begin{itemize}
            \item /org/Cinnamon 
            \item /com/Test
            \item /org/kde/kspread/sheets/3/cells/4/5
        \end{itemize}
        \item Implementuje interfejsy
        \item Implementuje sygnały i metody
        \item Wymaga określenia bus name
    \end{itemize}
\end{frame}
